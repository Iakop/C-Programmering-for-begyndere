%% Præsentation for C-programmering for begyndere
%% Lavet af Jacob Bechmann Pedersen og Jacob Skjødt Nielsen
%% For C undervisning i IDA 
%%
%% Theme: `DarkConsole'
%% Copyright (c) 2011-2017 Kazuki Maeda <kmaeda@kmaeda.net>
%% 
%% Distributable under the MIT License:
%% http://www.opensource.org/licenses/mit-license.php 

%% Preamble
\documentclass{beamer}

\usepackage{hyperref} % Add a link to your document
\usepackage{graphicx} % Add pictures to your document
\usepackage{listings} % Source code formatting and highlighting
\usepackage[utf8]{inputenc} % Gives UTF-8 encoded characters such as Æ, Ø, Å.

\usetheme{DarkConsole}
\title{C-Programmering for begyndere}
\subtitle{Del 1 - Hello World, Datatyper, Matematik og I/O}
\author{Jacob B. Pedersen\footnote{jacob.bp@mvb.net} og Jakob S. Nielsen}

%% Document
\begin{document}

\begin{frame}
	\begin{columns}
		\begin{column}
			\maketitle
		\end{column}
		\begin{column}
			\includegraphics{assets/The_C_Programming_Language_logo.png}
		\end{column}
	\end{columns}
\end{frame}

\begin{frame}{Indhold}
  \tableofcontents
\end{frame}

\section{Intro til C}

\begin{frame}{C's historie}
  Indsæt C's historie her.

  \pause

  The followings are mathematical lists.

  \begin{enumerate}
  \item Item $1$\pause
  \item Item $1+1$\pause
  \item Item $1+1+1$
  \end{enumerate}

  \pause

  \begin{itemize}
  \item Item $1+1+1+1$\pause
  \item Item $1+1+1+1+1$\pause
  \item Item $1+1+1+1+1+1$
  \end{itemize}
\end{frame}

\begin{frame}{Slide $1+1$}
  \alert{Get started in writing equations!!!}

  \begin{theorem}[Gaussian integral]
    The following integral is very well known:
    \begin{equation}
      \int_{-\infty}^\infty \mathrm{e}^{-x^2}\,\mathrm{d}x=\sqrt{\pi}.
    \end{equation}

  \end{theorem}
\end{frame}

\section{More Mathematical Story}
\begin{frame}{Slide $1+1+1$}
\end{frame}
\end{document}
