%% Præsentation for C-programmering for begyndere
%% Lavet af Jacob Bechmann Pedersen og Jacob Skjødt Nielsen
%% For C undervisning i IDA 
%%
%% Theme: `DarkConsole'
%% Copyright (c) 2011-2017 Kazuki Maeda <kmaeda@kmaeda.net>
%% 
%% Distributable under the MIT License:
%% http://www.opensource.org/licenses/mit-license.php 

%% Preamble
\documentclass{beamer}

\usepackage{hyperref} % Add a link to your document
\usepackage{graphicx} % Add pictures to your document
\usepackage{listings} % Source code formatting and highlighting
\usepackage[utf8]{inputenc} % Gives UTF-8 encoded characters such as Æ, Ø, Å.

%% Setting the C language type, for viewing pleasure:
\usepackage{listings}
\usepackage{color}

\definecolor{dkgreen}{rgb}{0,0.6,0}
\definecolor{gray}{rgb}{0.5,0.5,0.5}
\definecolor{mauve}{rgb}{0.58,0,0.82}

\lstset{frame=tb,
  language=C,
  aboveskip=3mm,
  belowskip=3mm,
  showstringspaces=false,
  columns=flexible,
  basicstyle={\small\ttfamily},
  numbers=left,
  numbersep=0pt,
  keywordsprefix={\#, \<},
  numberstyle=\tiny\color{gray},
  keywordstyle=\color{C_darkblue},
  commentstyle=\color{dkgreen},
  stringstyle=\color{C_lightblue},
  breaklines=true,
  breakatwhitespace=true,
  tabsize=3
}

\usetheme{C_Console}
\title{C-Programmering for begyndere}
\subtitle{Del 1 - Hello World, Datatyper, Matematik og I/O}
\author{Jacob B. Pedersen\footnote{jacob.bp@mvb.net} og Jakob S. Nielsen\footnote{indsæt mail her}}

%% Document
\begin{document}

\begin{frame}
	\maketitle
\end{frame}

\begin{frame}{Indhold}
	\tableofcontents
\end{frame}

\section{Jeres undervisere}
%%----------------------------------------------------------------------
\subsection{Hvem er vi?}

\begin{frame}{Hvem er vi?}
	\begin{columns}

		\begin{column}{0.5\textwidth}
		\begin{center}
			\includegraphics[width=0.4\textwidth]{assets/jacob_bp.png}
		\end{center}
		\begin{itemize}
		\item{Jacob Bechmann Pedersen}
			\begin{itemize}
			\item{Læser Elektronikingeniør}
			\item{6. Semester}
			\end{itemize}
		\item{Direkte fra HTX}
		\item{Startet egen virksomhed}
			\begin{itemize}
			\item{Bechmann \& Vang - Med fokus på elektroniske musikinstrumenter}
			\end{itemize}
		\item{Holder også Arduino workshops}
		\end{itemize}
		\end{column}
		
		\begin{column}{0.5\textwidth}
		\begin{center}
			\includegraphics[width=0.4\textwidth]{assets/jakob_sn.jpg}
		\end{center}
		\begin{itemize}
		\item{Jakob Skjødt Nielsen}
			\begin{itemize}
			\item{Læser Elektronikingeniør}
			\item{2. Semester}
			\end{itemize}
		\item{Elektronikfagtekniker ved B\&O}
		\item{Osv.}
		\item{Osv.}
		\end{itemize}
		\end{column}
		
	\end{columns}
\end{frame}

\section{Intro til C}
%%----------------------------------------------------------------------
\subsection{Hvad er C?}

\begin{frame}{Hvad er C?}
	\begin{columns}
	
		\begin{column}{0.5\textwidth}
		\begin{itemize}
		\item {C er letvægtigt og hurtigt!}
			\begin{itemize}
			\item{Fylder meget lidt plads}
			\item{Kører meget stærkt!}
			\end{itemize}
		\item {Kompatibelt og portabelt til de fleste systemer}
		\item {Kan bruges bredt!}
			\begin{itemize}
			\item {Kan gå dybt i detaljerne}
			\item {Eller behandle i overfalden}
			\end{itemize}
		\item{C er et compiled sprog}
			\begin{itemize}
			\item {Koden samles til 1'er og 0'er én gang}
			\item {Inden det køres, modsat interpreted}
			\end{itemize}
		\end{itemize}
		\end{column}
		
		\begin{column}{0.5\textwidth}
		\begin{center}
     		\includegraphics[width=0.7\textwidth]{assets/The_C_Programming_Language_logo.png}
     	\end{center}
		\end{column}
		
	\end{columns}
\end{frame}
%%----------------------------------------------------------------------
\subsection{C's historie}

\begin{frame}{C's historie}
	\begin{columns}
	
		\begin{column}{0.5\textwidth}
		\begin{itemize}
		\item {C begyndte som så meget andet i Bell Labs}
			%% Bell Labs er bl.a. ansvarlige for:
			%% Transistoren
			%% Laseren
			%% UNIX, C og C++
			\begin{itemize}
			\item{Skabt af Dennis Ritchie} %% Ham til venstre
			\item{Baseret på B af Ken Thompson} %% Ham til højre
			%% Basic Combined Programming Language
			\item{Udviklet til arbejdet på UNIX}
			\end{itemize}
		\item {Udviklet af programmører til programmører!}
		\end{itemize}
		\end{column}
		
		\begin{column}{0.5\textwidth}
		\begin{center}
     		\includegraphics[width=0.6\textwidth]{assets/Lucent_HQ.png} 					\break
     		\break
     		\includegraphics[width=0.6\textwidth]{assets/Ken_n_dennis.png}
     	\end{center}
		\end{column}
		
	\end{columns}
\end{frame}
%%---------------------------------------------------------------------

\subsection{Eksempler på brug af C}
\begin{frame}{Eksempler på brug af C}
	\begin{columns}
	
	\begin{column}{0.5\textwidth}
	\begin{itemize}
	\item{Kernen til dit Operativsystem er hovedsagligt skrevet i C}
	\item{C har inspireret mange sprog}
		\begin{itemize}
		\item{Swift}
		\item{Java}
		\item{PHP}
		\item{R}
		\end{itemize}
	\item{Compilers og interpreters er ofte skrevet i C!}
	\item{C bruges i indlejret elektronik i stor stil!}
	\end{itemize}
	\end{column}
	
	\begin{column}{0.5\textwidth}
	\begin{center}
		\includegraphics[width=0.7\textwidth]{assets/windows-kernel.png}
		\break
		\includegraphics[width=0.7\textwidth]{assets/C-derivatives.png}
	\end{center}
	\end{column}
	
	\end{columns}
\end{frame}

\section{Et C-programs opbygning}
%%----------------------------------------------------------------------
\subsection{Et C-programs opbygning}
\begin{frame}[fragile]{Et C-programs opbygning}
	\begin{columns}
	
	\begin{column}{0.5\textwidth}
	
	\begin{itemize}
	\item{Linje 3:}
		\begin{itemize}
		\item{\bf{main()} er & hovedfunktionen}
		\end{itemize}
	\end{itemize}
	
	\begin{itemize}
	\item{Linje 6:}
		\begin{itemize}
		\item{Funktioner "returnerer" & ofte status til sidst}
		\end{itemize}
	\end{itemize}
	
	\begin{itemize}
	\item{Linje 1:}
		\begin{itemize}
		\item{Libraries med standardkomponenter}
		\end{itemize}
	\end{itemize}
	
	\begin{itemize}
	\item{Linje 5:}
		\begin{itemize}
		\item{printf() er en af dem. & Den skriver til konsollen}
		\end{itemize}
	\end{itemize}
	
	\end{column}
	
	\begin{column}{0.4\textwidth}
	\begin{lstlisting}
	#include <stdio.h>

	int main(void)
	{
		printf("Hello world\n");
		return 0;
	}
	\end{lstlisting}
	\end{column}
	
	\end{columns}
	
	\begin{center}
	Output:
	\lstset{language=bash, numbers=none}
	\begin{lstlisting}
	Hello world
	\end{lstlisting}
	\end{center}

\end{frame}
%%----------------------------------------------------------------------
\begin{frame}[fragile]{Et C-programs opbygning}
	\begin{columns}
	\begin{small}
	\begin{column}{0.5\textwidth}
	\begin{itemize}
	\item{Linje 3:}
		\begin{itemize}
		\item{Man kan deklarere variable}
		\end{itemize}
	\end{itemize}
	
	\begin{itemize}
	\item{Linje 10:}
		\begin{itemize}
		\item{Og lægge dem sammen}
		\end{itemize}
		\begin{itemize}
		\item{I en ny variabel}
		\end{itemize}
	\end{itemize}
	
	\begin{itemize}
	\item{Linje 11:}
		\begin{itemize}
		\item{Skrive dem ud, & formateret}
		\end{itemize}
	\end{itemize}
	\end{column}
	\end{small}
	
	\begin{column}{0.4\textwidth}
	\begin{lstlisting}
	#include <stdio.h>
	
	int a;
	int b;	
	
	int main(void)
	{
		a = 2;
		b = 3;
		int c = a + b;
		printf("%d + %d = %d\n", a, b, c);
		return 0;
	}
	\end{lstlisting}
	\end{column}
	
	\end{columns}
	\begin{center}
	Output:
	\lstset{language=bash, numbers=none}
	\begin{lstlisting}
	2 + 3 = 5 
	\end{lstlisting}
	\end{center}
\end{frame}

\section{Øvelseseksempler}
%%----------------------------------------------------------------------
\subsection{Opsætning af CodeLite}
\begin{frame}{Opsætning af CodeLite}
	\begin{itemize}
	\item{Nu skal vi have opsat CodeLite}
	\begin{itemize}
		\item{\url{https://downloads.codelite.org/}}
		\end{itemize}
	\item{Hvad man kalder et IDE}
		\begin{itemize}
		\item{Integrated Development Environment}
		\end{itemize}
	\item{Det gør vi live!}
	\end{itemize}
\end{frame}

%%----------------------------------------------------------------------
\subsection{Eksempel 0 - Hello World}
\begin{frame}{Eksempel 0 - Hello World}
	\begin{itemize}
	\item{Det første program hedder Hello World}
	\item{Det er en klassiker inden for programmering}
		\begin{itemize}
		\item{Første program man laver, så der er hul igennem}
		\end{itemize}
	\item{\href{https://github.com/Iakop/C-Programmering-for-begyndere/tree/master/Del_1/Examples/1_Hello_World/}{../Del\_1/Examples/1\_Hello\_World/}}
	\item{Vi tager den i fællesskab!}
	\end{itemize}
\end{frame}

%%----------------------------------------------------------------------
\begin{frame}{Eksempel 0 - Hello World}[fragile]
	\begin{lstlisting}
	
	\end{lstlisting}
\end{frame}

\end{document}

%%\begin{frame}[fragile]
	%%\begin{lstlisting}
		%%#include <stdio.h>

		%%int main(void)
		%%{
		%%printf("Hello world\n");
		%%return 0;
		%%}
	%%\end{lstlisting}
%%\end{frame}