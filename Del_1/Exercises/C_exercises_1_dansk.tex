\documentclass[hidelinks]{article} % Class of the document. Article is used for most reports.

% Packages - Use as seen fit.
\usepackage{amsmath} % Advanced math typesetting
%\usepackage[ngerman]{babel} % Change hyphenation rules
\usepackage{hyperref} % Add a link to your document
\usepackage{graphicx} % Add pictures to your document
\usepackage{listings} % Source code formatting and highlighting
\usepackage[utf8]{inputenc} % Gives UTF-8 encoded characters such as Æ, Ø, Å.
\usepackage{a4wide} % Widens the formatting to fit on standard A4 pages.
\usepackage{fancyhdr} % For headers and footers
\usepackage{lastpage} % For getting the last page number.

% Title page info.
\title{C-programmering for begyndere - Del 1
 		\break
 		\break
		\large Øvelser} % Title of the document.
\date{d. 9. april 2018} % Date of the document.
\author{Jacob B. Pedersen \break \& \break Jakob S Nielsen} % My name, the author. 

\pagestyle{fancy}
\lhead{Jacob B. Pedersen \& Jakob S Nielsen}
\chead{2018-04-09}
\rhead{[Organisation]}
\cfoot{Page \thepage\ of \pageref{LastPage}}
\renewcommand{\headrulewidth}{0.5pt}
\renewcommand{\footrulewidth}{0.5pt}

\begin{document}

	%%\begin{titlepage}
	%%\pagenumbering{gobble} % Removes page numbering on front page.
  	%%\maketitle % Generates the title, author and date.
  	%%\end{titlepage}
  	%%\newpage % Generates a new page.
  	%%\begin{titlepage}
  		%%\renewcommand{\contentsname}{Indhold} % Change the title of Contents.
  		%%\tableofcontents % Generates a table of contents.
  	%%\end{titlepage}
  	%%\newpage % Generates a new page.
  	\pagenumbering{arabic} % Sætter sidenummerering til igen, med arabiske tal.

	\section{Skriv en Fødselsårsomregner}
Uden at tage hensyn til skudår, skriv et program, hvor en variabel rerpæsenterer dit fødselsår, og et andet repræsenterer omregningen fra år til dage.

De kunne evt. hedde noget retvisende som 
\begin{verbatim}birthyear
\end{verbatim}
og
\begin{verbatim}yearAsDays
\end{verbatim}

Få programmet til at printe resultatet til konsollen. Eventuelt med en lille besked inden:

\begin{verbatim}Your age in days is: <days>
\end{verbatim}

	\subsection{scanf() input}
Få dit nye program til at tage input gennem konsollen, og lægge dette input i dine variable, så den kan omregnes.

	\section{Omregn kropsmål til BMI}
Skriv et program, der tager input i form af kropsparametrene til en BMI udregning. De er som følger:
\begin{verbatim}[vægt i kg]/[højde i m]^2
\end{verbatim}

Brug scanf() til at behandle inputtet. Og skriv det derefter ud med printf().

	\section{Konverter mellem enheder}
Du står og har brug for at kende et centimetermål i tommer. Derfor beslutter du dig for at skrive et program, der kan konvertere mellem disse to enheder:

1 cm er lig med 0.3937008 tommer.

Der bliver altså brug for en float repræsentation af tallene.
Prøv først at skrive målet direkte ind i programmet.

	\subsection{Tag input}
Udvid nu programmet til at tage mod input vha. scanf().

	\subsection{Beregn flere enheder}
Udvid programmet med nogle flere enheder at konvertere imellem. lidt inspiration:

\begin{itemize}
	\item{liter til oz.}
	\item{spiseskefuld til teskefuld}
	\item{kg til pund}
	\item{DKK til USD}
	\item{Celsius til Fahrenheit}
	\item{km/t til mph}
\end{itemize}

Brugeren skal indtaste et input til hver af målene.

	\section{Præsentér dig selv}
Du vil gerne introducere dig over for nye kolleger/klassekammerater på en kreativ måde, så du skriver et program, de kan køre for at høre om dig.

Du vil programmere dit navn end som en
\begin{verbatim} char array[]
\end{verbatim}
Så hele navnet kan printes ud med printf() og format specifieren \%s.
Lidt ligesom:
\begin{verbatim}
	Hej, mit navn er [navn].
	Mine interesser er bl.a. [interesse1] og [interesse2].
	Jeg studerer/arbejder på [sted], som [studie/erhverv].
	
	Rart at møde dig!
\end{verbatim}
	\subsection{Tag input}
	Implementér nu input vha. scanf, så du kan skrive dit navn ind, og så få linjen ud med det indsat. Du kan også prompte brugeren for at skulle trykke [enter] for at komme videre i teksten.
\end{document}


