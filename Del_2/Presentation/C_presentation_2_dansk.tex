%% Præsentation for C-programmering for begyndere
%% Lavet af Jacob Bechmann Pedersen og Jacob Skjødt Nielsen
%% For C undervisning i IDA 
%%
%% Theme: `DarkConsole'
%% Copyright (c) 2011-2017 Kazuki Maeda <kmaeda@kmaeda.net>
%% Distributable under the MIT License:
%% http://www.opensource.org/licenses/mit-license.php 

%% Preamble
\documentclass{beamer}

\usepackage{hyperref} % Add a link to your document
\usepackage{graphicx} % Add pictures to your document
\usepackage{listings} % Source code formatting and highlighting
\usepackage[utf8]{inputenc} % Gives UTF-8 encoded characters such as Æ, Ø, Å.

%% Setting the C language type, for viewing pleasure:
\usepackage{listings}
\usepackage{color}

\definecolor{link}{HTML}{CF55E3}
\definecolor{dkgreen}{rgb}{0,0.6,0}
\definecolor{gray}{rgb}{0.5,0.5,0.5}

\lstset{frame=tb,
  inputencoding=utf8,
  language=C,
  aboveskip=3mm,
  belowskip=3mm,
  showstringspaces=false,
  columns=flexible,
  basicstyle={\small\ttfamily},
  numbers=left,
  numbersep=0pt,
  keywordsprefix={\#, \<},
  numberstyle=\tiny\color{gray},
  keywordstyle=\color{C_darkblue},
  commentstyle=\color{dkgreen},
  stringstyle=\color{C_lightblue},
  breaklines=true,
  breakatwhitespace=true,
  tabsize=3,
  extendedchars=true,
  literate={æ}{{\ae}}1 {ø}{{\o}}1 {å}{{\r a}}1 {Æ}{{\AE}}1 {Ø}{{\O}}1 {Å}{{\r A}}1,
}

\usetheme{C_Console}
\title{C-Programmering for begyndere}
\date{16. april 2018}
\subtitle{Del 2 - Operatorer, conditionals, loops og forgreninger}
\author{Jacob B. Pedersen\footnote{jacob.bp@mvb.net} og Jakob S. Nielsen\footnote{jakob990@gmail.com}}

%% Document
\begin{document}

\begin{frame}
	\maketitle
\end{frame}

\begin{frame}{Indhold}
	\tableofcontents
\end{frame}

\section{Repetition}
%%----------------------------------------------------------------------
\subsection{Hvad lavede vi sidste gang?}

\begin{frame}{Hvad lavede vi sidste gang?}
	\begin{itemize}
		\item{Vi fik sat vores første program op}
		\begin{itemize}
			\item{Hello World}
		\end{itemize}
		\item{Stiftede bekendtskab med et par datatyper}
		\begin{itemize}
			\item{{\color{C_darkblue}int}, {\color{C_darkblue}float}, {\color{C_darkblue}char}, {\color{C_darkblue}bool}}
		\end{itemize}
		\item{Vi tog input, og skrev ud på konsollen}
		\begin{itemize}
			\item{{\color{C_darkblue}scanf}() og {\color{C_darkblue}printf}() fra {\color{C_lightblue}stdio.h}}
		\end{itemize}
	\end{itemize}
\end{frame}

%%----------------------------------------------------------------------
\subsection{Hello World}

\begin{frame}[fragile]{Hello World}
	\begin{itemize}
		\item{Begynderprogrammet}
		\item{Indeholdt {\color{C_darkblue}main}(), {\color{C_lightblue}stdio.h}, {\color{C_darkblue}printf}() og en {\color{C_darkblue}return} 0}
	\end{itemize}
		\begin{lstlisting}
		#include <stdio.h>

		int main(void)
		{
			printf("Hello world\n");
			return 0;
		}
		\end{lstlisting}
\end{frame}

%%----------------------------------------------------------------------
\subsection{Datatyper}

\begin{frame}[fragile]{Datatyper}
	\begin{itemize}
		\item{Et lille {\color{dkgreen}cheatsheet} over datatyperne i C:}
	\end{itemize}
		\begin{lstlisting}
		int i; // Integer/heltal - *mindst* 16 bits
				// Fra [-32767 til +32767]
		float f; // Floating point decimal - 32 bits
				// *mindst* 6 betydende cifre
		char ch; // Character - *mindst* 8 bits
				// ASCII dækker fra 0-128 forskellige tegn
		bool b; // Boolean - true/false - *mindst* 8 bits
				// Kun 0 anses som Boolsk "false" ved tilskrivning
				// Skal hentes med #include <stdbool.h>
		\end{lstlisting}
\end{frame}

%%----------------------------------------------------------------------

\begin{frame}[fragile]{Datatyper}
	\begin{itemize}
		\item{Lidt flere udvidede datatyper:}
	\end{itemize}
		\begin{lstlisting}
		unsigned int ui; // Integer/heltal - *mindst* 16 bits
				// [0 til +65535]
		short int f; // Short integer - *mindst* 16 bits
				// [−32767 til +32767]
		long int li; // Long integer - *mindst* 32 bits
				// [−2147483647 til +2147483647]
		double db; // Double precision float - *mindst* 64 bits
				// *mindst* 10 betydende cifre
		long double ldb; // Double precision float - *mindst* 128 bits - *mindst* 10 betydende cifre
		\end{lstlisting}
\end{frame}

%%----------------------------------------------------------------------
\subsection{Input og output}

\begin{frame}[fragile]{Input og output}
	\begin{itemize}
		\item{Til input og output bruges {\color{C_lightblue}\#include <stdio.h>}}
		\begin{itemize}
			\item{{\color{C_darkblue}scanf}() tager input, eks:}
			\begin{lstlisting}
	scanf("%d %d", &variable1, &variable2);
			\end{lstlisting}
			\item{{\color{C_darkblue}printf}() skriver output, eks:}
			\begin{lstlisting}
	printf("Variable 1: %d Variable 2: %d", variable1, variable2);
			\end{lstlisting}
		\end{itemize}
		\item{Vi bruger {\color{dkgreen}format specifiers} til at fortælle {\color{C_darkblue}printf}() og {\color{C_darkblue}scanf}() hvilken datatype det skal fortolkes som}
		\begin{lstlisting}
	"%d" // Integer format specifier - printer/scanner heltal
	"%f" // Float format specifier - printer/scanner kommatal
		"%.2f" // Float med 2 decimaltal
	"%c" // Char format specifier - printer/scanner skrifttegn
	"%s" // String format specifier - printer/scanner skriftstrenge
		\end{lstlisting}
	\end{itemize}
\end{frame}

\section{I dag}
%%----------------------------------------------------------------------
\subsection{Dagens program}

\begin{frame}[fragile]{Dagens program}
	\begin{itemize}
		\item{I dag vil vi kigger på følgende:}
		\item{Operatorer}
		\begin{itemize}
			\item{{\color{dkgreen}<}, {\color{dkgreen}>}, {\color{dkgreen}==}, {\color{dkgreen}!=}, {\color{dkgreen}\&\&}, {\color{dkgreen}||}, {\color{dkgreen}++}, {\color{dkgreen}--}}
		\end{itemize}
		\item{If-else udtryk}
		\begin{itemize}
			\item{{\color{C_darkblue}if}(), {\color{C_darkblue}else if}(), {\color{C_darkblue}else}()}
		\end{itemize}
		\item{Løkker}
		\begin{itemize}
			\item{{\color{C_darkblue}while}(), {\color{C_darkblue}for}()}
		\end{itemize}
		\item{Switch-cases}
		\begin{itemize}
			\item{{\color{C_darkblue}switch}()}
		\end{itemize}
	\end{itemize}
\end{frame}

%%----------------------------------------------------------------------
\subsection{Operatorer}

\begin{frame}[fragile]{Operatorer}
	\begin{itemize}
		\item{Vi kender allerede de basale, vi brugte til udregninger mm.}
		\begin{itemize}
			\item{{\color{dkgreen}+}, {\color{dkgreen}-}, {\color{dkgreen}*}, {\color{dkgreen}/}}
			\item{{\color{dkgreen}=}, {\color{dkgreen}!}}
		\end{itemize}
		\item{Men vi har også brug for at sammenligne værdier}
		\item{Vi kender det f.eks. fra ligninger og uligheder}
		\begin{itemize}
			\item{{\color{dkgreen}>}, {\color{dkgreen}<}, {\color{dkgreen}>=}, {\color{dkgreen}<=}}
		\end{itemize}
		\item{Når C ser dette:}
		\begin{lstlisting}
		12 > 10
		\end{lstlisting}
		\item{Erstattes det med en Boolsk værdi (sandt eller falsk):}
		\begin{lstlisting}
		true
		\end{lstlisting}
	\end{itemize}
\end{frame}

%%----------------------------------------------------------------------

\begin{frame}[fragile]{Operatorer}
	\begin{itemize}
		\item{Hvilke operatorer har vi så?}
		\begin{lstlisting}
		== // Lighed - er værdierne lig hinanden?
		!= // Ikke lig - er værdierne forskellige?
		> // Større - Er værdien til venstre større?
		< // Mindre - Er værdien til venstre mindre?
		>= // Større/lig - Er værdien til venstre større/lig?
		<= // Mindrel/lig - Er værdien til venstre mindre/lig?
		|| // Or - Bruges til at samle udtryk (er A > B ELLER er C > B)
		&& // And - Bruges til at samle udtryk (er A > B OG er C > B)
		\end{lstlisting}
	\end{itemize}
\end{frame}
%%----------------------------------------------------------------------

\begin{frame}[fragile]{Operatorer}
	\begin{itemize}
		\item{Foruden sammenligningerne er der også et ekstra sæt fikse operatorer til aritmetik og tilskrivning:}
		\begin{lstlisting}
		+= // Tilskriver venstrehåndsværdien dens egen + højrehåndsværdien
		-= // Tilskriver venstrehåndsværdien dens egen - højrehåndsværdien
		++ // Inkrementerer en variabel eks:
			i++; // Evaluerer i, og inkrementerer med én efter
			++i; // Inkrementerer først, og evaluerer herefter i
		-- // Dekrementerer en variabel på samme måde
		\end{lstlisting}
	\end{itemize}
\end{frame}

%%----------------------------------------------------------------------
\subsection{If-Else udtryk}

\begin{frame}[fragile]{If-Else udtryk}
	\begin{itemize}
		\item{Programmer skal også kunne noget intelligent}
		\item{De skal have indbygget logik!}
		\item{Den mest basale måde, er at bruge conditionals:}
		\begin{itemize}
			\item{{\color{C_darkblue}if}()}
			\item{{\color{C_darkblue}else if}()}
			\item{{\color{C_darkblue}else}()}
		\end{itemize}
	\end{itemize}
\end{frame}

%%----------------------------------------------------------------------

\begin{frame}[fragile]{If-Else udtryk}
	\begin{itemize}
		\item{Hvis vi skal illustrere det:}
		\begin{lstlisting}
		if ([sandt/falsk udtryk])
		{
			handling();
		}
		else if ([andet udtryk])
		{
			andenHandling();
		}
		else
		{
			tredjeHandling();
		}
		\end{lstlisting}
	\end{itemize}
\end{frame}

%%----------------------------------------------------------------------

\subsection{Løkker}

\begin{frame}[fragile]{Løkker}
	\begin{itemize}
		\item{En anden vigtig kontrolstruktur er loopet. Løkken.}
		\item{Hvor et stykke af programmet gentager sig, mens betingelser er opfyldte}
		\item{I C er der to basale løkketyper:}
		\begin{itemize}
			\item{{\color{C_darkblue}while}() og {\color{C_darkblue}for}()}
		\end{itemize}
	\end{itemize}
\end{frame}

%%----------------------------------------------------------------------

\begin{frame}[fragile]{Løkker}
	\begin{itemize}
		\item{En {\color{C_darkblue}while}() fungerer således:}
		\begin{lstlisting}
		while([betingelse])
		{
			handling();
		}
		\end{lstlisting}
		\item{En {\color{C_darkblue}for}() fungerer således:}
		\begin{lstlisting}
		for(int j = 0; j < 5; j++)
		{
			handling();
		}
		\end{lstlisting}
	\end{itemize}
\end{frame}

%%----------------------------------------------------------------------

\begin{frame}[fragile]{Løkker}
	\begin{itemize}
		\item{herudover eksisterer to brugbare nøgleord:}
		\begin{lstlisting}
		continue; // Springer videre til næste loopiteration
		break; // Bryder ud af loop (og scope)
		\end{lstlisting}
	\end{itemize}
\end{frame}

%%----------------------------------------------------------------------

\subsection{Switch-cases}

\begin{frame}[fragile]{Switch-cases}
	\begin{itemize}
		\item{Sidste ting i dag: {\color{dkgreen}Switch-case}}
		\item{Før forgrenede vi vha. if-else udtryk}
		\item{{\color{C_darkblue}switch}() kigger dog ikke på en evaluering}
		\begin{itemize}
			\item{Den betragter værdierne dens input-variabel har}
			\item{Og sender programmet til et sted i koden, på baggrund deraf}
			\item{Det virker godt til forgrening på baggrund af brugerinput!}
		\end{itemize}
	\end{itemize}
\end{frame}

%%----------------------------------------------------------------------

\begin{frame}[fragile]{Switch-cases}
	\begin{itemize}
		\item{En {\color{C_darkblue}switch}() fungerer således:}
		\begin{lstlisting}
		switch(variabel)
		{
			case 1:
			handling();
			break;
			
			case 2:
			andenHandling();
			break;
			
			default:
			tredjeHandling();
			break;
		}
		\end{lstlisting}
	\end{itemize}
\end{frame}

\section{Kreative Opgaver}
%%----------------------------------------------------------------------
\begin{frame}{Kreative Opgaver}
	\begin{itemize}
	\item{Det var det for nu!}
	\item{Der ligger som sidst kreative opgaver tilgængelige:}
		\begin{itemize}
		\item{\color{link}\href{https://github.com/Iakop/C-Programmering-for-begyndere/tree/master/Del_2/Exercises/C_exercises_2_dansk.pdf}{../Del\_2/Exercises/C\_exercises\_2\_dansk.pdf}}
		\end{itemize}
	\item{Der er hjælp at hente her på workshoppen}
	\item{God arbejdslyst! - Happy Hacking!}
	\end{itemize}
\end{frame}



\end{document}

%%\begin{frame}[fragile]
	%%\begin{lstlisting}
		%%#include <stdio.h>

		%%int main(void)
		%%{
		%%printf("Hello world\n");
		%%return 0;
		%%}
	%%\end{lstlisting}
%%\end{frame}