\documentclass[hidelinks]{article} % Class of the document. Article is used for most reports.

% Packages - Use as seen fit.
\usepackage{amsmath} % Advanced math typesetting
%\usepackage[ngerman]{babel} % Change hyphenation rules
\usepackage{hyperref} % Add a link to your document
\usepackage{graphicx} % Add pictures to your document
\usepackage{listings} % Source code formatting and highlighting
\usepackage[utf8]{inputenc} % Gives UTF-8 encoded characters such as Æ, Ø, Å.
\usepackage{a4wide} % Widens the formatting to fit on standard A4 pages.
\usepackage{fancyhdr} % For headers and footers
\usepackage{lastpage} % For getting the last page number.

%% Setting the C language type, for viewing pleasure:
\usepackage{listings}
\usepackage{color}

\definecolor{link}{HTML}{CF55E3}
\definecolor{dkgreen}{rgb}{0,0.6,0}
\definecolor{gray}{rgb}{0.5,0.5,0.5}

\lstset{frame=tb,
  inputencoding=utf8,
  language=C,
  aboveskip=3mm,
  belowskip=3mm,
  showstringspaces=false,
  columns=flexible,
  basicstyle={\small\ttfamily},
  numbers=none,
  numbersep=0pt,
  keywordsprefix={\#, \<},
  %%numberstyle=\tiny\color{gray},
  keywordstyle=\textbf,
  commentstyle=\textit,
  stringstyle=\textbf,
  breaklines=true,
  breakatwhitespace=true,
  tabsize=3,
  extendedchars=true,
  literate={æ}{{\ae}}1 {ø}{{\o}}1 {å}{{\r a}}1 {Æ}{{\AE}}1 {Ø}{{\O}}1 {Å}{{\r A}}1,
}

% Title page info.
\title{C-programmering for begyndere - Del 1
 		\break
 		\break
		\large Øvelser} % Title of the document.
\date{d. 16. april 2018} % Date of the document.
\author{Jacob B. Pedersen \break \& \break Jakob S Nielsen} % My name, the author. 

\pagestyle{fancy}
\lhead{Jacob B. Pedersen \& Jakob S Nielsen}
\chead{2018-04-16}
\rhead{IDA AU Herning}
\cfoot{Page \thepage\ of \pageref{LastPage}}
\renewcommand{\headrulewidth}{0.5pt}
\renewcommand{\footrulewidth}{0.5pt}

\begin{document}

	%%\begin{titlepage}
	%%\pagenumbering{gobble} % Removes page numbering on front page.
  	%%\maketitle % Generates the title, author and date.
  	%%\end{titlepage}
  	%%\newpage % Generates a new page.
  	%%\begin{titlepage}
  		%%\renewcommand{\contentsname}{Indhold} % Change the title of Contents.
  		%%\tableofcontents % Generates a table of contents.
  	%%\end{titlepage}
  	%%\newpage % Generates a new page.
  	\pagenumbering{arabic} % Sætter sidenummerering til igen, med arabiske tal.

\section{Giv din maskine fordomme!}	
Skriv et program, der tage brugerinput i form af din alder i år, ved hjælp af \lstinline{scanf()}
Husk at prompte brugeren først, a la:
\begin{lstlisting}
Hello, nice to meet you! Tell me, how old are you?
\end{lstlisting}
Hvorefter den skal svare med en besked, der ændrer karakter efter om du er over eller under 25:
\begin{lstlisting}
You’re just 23! Wow, youngster are you old enough to be here?
\end{lstlisting}
eller
\begin{lstlisting}
Oh you’re 28? Didn’t know you were an old geezer
\end{lstlisting}
Vær kreativ!

\subsection{Forskellige udfald}
Skriv et forskelligt udfald for hvert årti. Der skal et par if statements til:
\begin{lstlisting}
if(age > 10 && age < 20)
{
	printf("Response");
}
\end{lstlisting}
Kan feks. Være for ranget 10-20.

\subsection{Forskellige udfald}
Der går rygter om at man også kan hacke en løsning sammen vha. switch cases, og det faktum at heltalsdivision med 10 runder ned. Kunne det her måske virke for hvert range?:
\begin{lstlisting}
switch(age/10)
{
	case 0:
		printf("Response");
		break;
	case 1:
		printf("Another response");
		break;
	...
	...
	...
}
\end{lstlisting}

\section{Skriv en Fødselsårsomregner}
	
\end{document}


